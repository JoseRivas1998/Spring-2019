\documentclass[12pt]{article}

\usepackage{amsmath}
\usepackage{amssymb}
\usepackage{amsthm}

\begin{document}
José de Jesus Rodriguez Rivas

MATH 300-01

\newcommand{\contradiction}{\Rightarrow\Leftarrow}

\begin{center}
Proof Portfolio
\end{center}
\section{Proof By Induction}

Show that
$$\sum_{i = 1}^{n} i! \cdot i = (n + 1)! - 1$$

\begin{proof}
By Induction:

Base Case: $n = 1$

LHS: $\sum_{i=1}^{1} i! \cdot i = 1! \cdot 1 = 1$

RHS: $(1 +1)! - 1 = 2! - 1 = 2 - 1 = 1$

$\text{LHS} = \text{RHS}$

Inductive Step: 

Assume 
$$\sum_{i = 1}^{n} i! \cdot i = (n + 1)! - 1$$

Show 
$$\sum_{i = 1}^{n+1} i! \cdot i = (n + 2)! - 1$$
\begin{align*}
\sum_{i = 1}^{n+1} i! \cdot i &= \sum_{i = 1}^{n} i! \cdot i + (n + 1)! (n+1)\\
&= (n+1)!-1 + (n + 1)!(n+1)\text{ By Induction Hyp.}\\
&= (n+1)! + (n+1)!(n+1)-1\\
&= (n+1)!(1 + (n+1)) - 1\\
&= (n+1)!(n+2)-1\\
&= (n+2)! - 1
\end{align*}
\end{proof}

\newpage
\begin{flushleft}
Prove that if $h > -1$, then $1 + nh \leq (h+1)^n$ for all non-negative integers $n$.
\end{flushleft}
\begin{proof}
By Induction.

Base Case:
$$n = 0$$
$$LHS = 1 + 0\cdot h = 1$$
$$RHS = (h+1)^0 = 1$$
$$LHS \leq RHS$$

Inductive Step:

Assume $1 + kh \leq (1+h)^k$, for $k = n$

Show $1 + h(k + 1) \leq (1+h)^{k+1}$

RHS:
$$(1+h)^{k+1} = (1+h)(1+h)^k$$
By Induction Hypothesis:
$$(1+h)^{k+1} \geq (1+h)(1+nh)$$
$$(1+h)^{k+1} \geq 1+nh+h+nh^2$$
$$(1+h)^{k+1} \geq 1+h(n+1)+nh^2$$

If LHS $\leq$ RHS, then $\text{RHS} - \text{LHS} \geq 0$. 

Lowest possible RHS = $1+h(n+1)+nh^2$
$$LHS - RHS = 1 + h(n + 1) - 1+h(n+1)+nh^2 = nh^2$$

Making $nh^2$ the lowest value for $\text{RHS} - \text{LHS}$.

Since $n \geq 0$ and $h^2 \geq 0$, then  $\text{RHS} - \text{LHS} \geq 0$


$$\therefore 1 + h(n + 1) \leq (1+h)^{k+1}$$

\end{proof}

\newpage
\section{Direct Proof}

Prove:
$$\binom{2n}{n} = \frac{2^n \cdot (2n-1)!!}{n!}$$
\begin{proof}
Direct.

First, show that $(2n-1)!! \cdot 2^n \cdot n! = (2n)!$
\begin{align*}
(2n-1)!! \cdot 2^n \cdot n! &= [(2n-1)(2n-3)...1][2(n)2(n-1)2(n-2)...2(1)]\\
&= [(2n-1)(2n-3)...1][(2n)(2n-2)(2n-4)...2]\\
&= (2n)(2n-1)(2n-2)(2n-3)(2n-4)...1\\
&= (2n)!
\end{align*}
\begin{align*}
\binom{2n}{n} &= \frac{(2n)!}{n!(2n - n)!}\\
&= \frac{(2n)!}{n! \cdot n!}
\end{align*}

By Above Derivation:
\begin{align*}
\binom{2n}{n} &= \frac{n! \cdot 2^n \cdot (2n-1)!!}{n! \cdot n!}\\
&= \frac{2^n \cdot (2n-1)!!}{n!}
\end{align*}

\end{proof}

\newpage
Show that
$$\sum_{r=0}^n r \binom{n}{r} = n\cdot 2^{n-1}$$

\begin{proof}
By Binomial Theorem Corollary:
$$\sum_{r=0}^n \binom{n}{r}= 2^n$$
$\binom{n}{r}$ Represents the number of subsets of size $r$ of $\{ 1, 2, ..., n\}$. The Binomial Theorem Corollary is saying that there are $2^n$ of these subsets. So, the sum of the sizes of the subsets is
$$\sum_{r=1}^n\binom{n}{r}\cdot r\text{, }r\text{ starts at 1 because when }r=0\text{, the size of the subset is also 0.}$$
Each element is in $2^{n-1}$ subsets, and so adds $2^{n-1}$ to the total sum, making the total sum: 
$$n2^{n-1}$$
\end{proof}

\newpage
\section{Proof By Contradiction}

Show that any simple, connected graph with 31 edges and 12 vertices is not planar.

\begin{proof}
Assume to the contrary that the graph is planar.

If we let $r$ be the number of regions in the graph, $p$ be the number of vertices in the graph, and $q$ be the number of edges.

By Euler's Formula:
\begin{align*}
r &= 2 - p + q\\
&= 2 - 12 + 31\\
&= 21
\end{align*}

So there are 21 regions in the graph.

The minimum degree per region in this graph is 3, so the minimum sum of the degrees of all the regions is
$$3r = 3 \cdot 21 = 63$$

By the Second Handshaking Lemma:
$$2q = \text{the sum of the degree of all the regions.}$$
$$2q = 2 \cdot 31$$
$$2q = 62$$

So the sum of the degrees must be at least 63, but is equal to 62. $\contradiction$

\end{proof}
\newpage
\begin{flushleft}
Show that $G$ is a simple graph with $p \geq 2$ vertices and $\text{deg}(v)\geq\frac{p - 1}{2}$ for each vertex $v$ in $G$. Prove $G$ is connected.
\end{flushleft}

\begin{proof}
Assume to the contrary that $G$ has 2 distinct connected components $C_1$ and $C_2$.

$C_1$ must have a vertex $v$, and at least $\frac{p-1}{2}$ other vertices, since they are connected to $v$. 

$$\therefore C_1 \text{ has } 1 + \frac{p-1}{2} = \frac{p+1}{2} \text{ vertices.}$$

A similar argument can be made for $C_2$ having $\frac{p+1}{2}$ vertices.

Since $C_1$ and $C_2$ are distinct components that make up $G$, the sum of the number of vertices in $C_1$ and $C_2$ must equal the number of vertices in $G$.
$$\therefore p = \frac{p+1}{2} + \frac{p + 1}{2}$$
$$p = p + 1 \contradiction$$

\end{proof}

\end{document}
